\documentclass[ngerman,a4paper,order=firstname]{mathscript}
\usepackage{mathoperators}

\usepackage{xcolor}
% % % local commands
\DeclareMathOperator{\Ad}{Ad}				% Adjoint
\DeclareMathOperator{\PSL}{PSL} 			% projective linear group 
\newcommand{\with}{\text{ with }}
\newcommand{\nd}{\text{ and }}
\renewcommand{\rhd}{\triangleright}
\renewcommand{\lhd}{\triangleleft} 			% normal subgroups
\DeclareMathOperator{\Set}{Set}				% Category of sets
\DeclareMathOperator{\Vect}{Vect}			% Category of vector spaces
\DeclareMathOperator{\Grp}{Grp}				% Category of groups
\DeclareMathOperator{\Mod}{Mod}				% Cat of moduls
\DeclareMathOperator{\Ann}{Ann}				% annihilator
\newcommand{\Circlearrowleft}{\rotatebox{180}{$\circlearrowright$}}
\DeclareMathOperator{\Cl}{Cl}				% conjugation class of something.
\DeclareMathOperator{\Iso}{Iso}

% % % % Model theory
\newcommand{\LL}{\mathcal L}
\newcommand{\RR}{\mathcal R}
\newcommand{\FF}{\mathcal F}
\newcommand{\MM}{\mathcal M} 		
\newcommand{\NN}{\mathcal N}
\newcommand{\CC}{\mathcal C}
\newcommand{\arity}{\lambda}

% % % % better \le and \ge
\renewcommand{\ge}{\geqslant}
\renewcommand{\le}{\leqslant}

\DeclareMathOperator{\group}{group}
\DeclareMathOperator{\ring}{ring}

% % % % color stuff
\newcommand{\selfnote}[1]{\textcolor{gray}{#1}}

% get this stupid arrows:
%\usepackage{mathabx,graphicx}  % ---> add to mathoperators
%\def\Circlearrowleft{\ensuremath{%
%		\rotatebox[origin=c]{180}{$\circlearrowleft$}}}
%\def\Circlearrowright{\ensuremath{%
%		\rotatebox[origin=c]{180}{$\circlearrowright$}}}
%\def\CircleArrowleft{\ensuremath{%
%		\reflectbox{\rotatebox[origin=c]{180}{$\circlearrowleft$}}}}
%\def\CircleArrowright{\ensuremath{%
%		\reflectbox{\rotatebox[origin=c]{180}{$\circlearrowright$}}}}
%\begin{document}
%	\Huge
%	$\circlearrowleft \circlearrowright $
%	
%	$\Circlearrowleft \Circlearrowright $
%	
%	$\CircleArrowleft \CircleArrowright $

% % % local packages
\usepackage{braids}

\newlist{remarkenum}{enumerate}{1}
\setlist[remarkenum]{label=(\alph*),ref=\theremark~(\alph*)}
\crefalias{remarkenumi}{remark}

\newlist{propenum}{enumerate}{1}
\setlist[propenum]{label=(\alph*),ref=\theproposition~(\alph*)}
\crefalias{propenumi}{proposition}

\newlist{expenum}{enumerate}{1}
\setlist[expenum]{label=(\alph*),ref=\theexample~(\alph*)}
\crefalias{expenumi}{example}

\newlist{lemmaenum}{enumerate}{1}
\setlist[lemmaenum]{label=(\alph*),ref=\thelemma~(\alph*)}
\crefalias{lemmaenumi}{lemma}

\newlist{defenum}{enumerate}{1}
\setlist[defenum]{label=(\roman*),ref=\thedefinition~(\roman*)}
\crefalias{defenumi}{definition}

\title{\textbf{WIA - Modelltheorie Bewerteter Körper}}
\author{Dozent: Prof. Dr. \person{Arno Fehm}}

\begin{document}
\pagenumbering{roman}
\pagestyle{plain}

\maketitle

\hypertarget{tocpage}{}
\tableofcontents
\bookmark[dest=tocpage,level=1]{Inhaltsverzeichnis}

\pagebreak
\pagenumbering{arabic}
\pagestyle{fancy}

%\chapter*{Vorwort}
%Dieser Kurs findet im Rahmen des wissenschaftlichen Arbeiten im Master statt. Und im Zuge der Corona Sache auch nur als Reading-Kurs mit Chats im Matrix System. Dazu werden Fragen zum Vorlesungsskript, Übungsaufgaben besprochen. Angedacht ist eigentlich ein Seminarteil, der bisher noch ungeregelt bleibt.

Ja, dann viel Spass bei \ital{Modelltheorie bewerteter Körper}
\chapter{Grundlagen}
\section{Körper}
\subsection{Körpererweiterungen}

\subsection{Galoistheorie}

\subsection{Ringerweiterungen}
\section{Modelltheorie}
\subsection{Spache und Strukturen}
Literatur
\begin{definition}[Sprache]
	Eine \begriff{Sprache} $\LL$ ist Quadrupel $\LL = (\FF,\RR, \CC, \arity)$, wobei $\FF,\RR, \CC$ Mengen sind und $\arity \colon \FF \cup \RR \to \N$.
	\begin{itemize}
		\item Elemente von $\FF,\RR,\CC$ nennen wir Funktionssymbole, Relationssymbole und Konstantensymbole (auch -zeichen)
		\item $\arity(f)$ bzw. $\arity(R)$ gibt die \begriff{Stelligkeit} von $f$ bzw. $R$ an
		\item Kardinalität  von $\LL$ ist $\abs{\LL} := \abs{\FF \cup \RR \cup \CC}$
	\end{itemize}
\end{definition}
\begin{example}
	\begin{itemize}
		\item \emph{Sprache der Gruppen} ist $\LL_{\group} = \set{\ast, \cdot^{-1}}$, wobei $\ast$ ist zweistellig und $\cdot^{-1}$ is einstelliges Funktionssymbol.
		\item \emph{Sprache der Ringe} ist $\LL_{\ring} = \set{+, \cdot, 0, 1}$, dabei sind $+$ und $\cdot$ zweistellig Funktionssymbole und 0,1 Konstantensymbole.
	\end{itemize}
\end{example}
\begin{definition}[$\LL$-Struktur]
	\proplbl{def_1_2_3}
	Sei $\LL = (\FF,\RR, \CC, \arity)$ eine \emph{Sprache}. Eine \begriff{$\LL$-Struktur} ist ein Tupel
	\begin{align*}
		\MM = (M, (f^{\MM})_{f \in \FF}, (R^{\MM})_{R \in \RR}, (c^{\MM})_{c \in \CC}),
	\end{align*}
	wobei
	\begin{itemize}
		\item $M$ \emph{nichtleere} Menge
		\item $f^{\MM}\colon M^{\arity(f)} \to M$
		\item $R^{\MM} \subseteq M^{\arity(R)}$
		\item $c^{\MM} \in M$
	\end{itemize}
	die \emph{Interpretation} der entsprechenden Symbole sind.
	\item \emph{Kardinalität} von $\MM$ ist $\abs{\MM} = \abs{M}$
\end{definition}
\begin{*remark}
	\selfnote{Wir werden lieber $R^{\MM}(a)$ für $\underline{a} \in R^{\MM}$ schreiben. ($\underline{a} \in M^{\arity(R)}$) Hierbei ist $\underline{a} = (a_1, \dots, a_n)$!}
\end{*remark}
\begin{example}
	Jeder Ring ist auf natürliche Weise eine $\LL_{\ring}$-Struktur.
\end{example}
\begin{definition}[Unterstruktur]
	Sei $\MM$ eine $\LL$-Struktur. Eine $\LL$-Struktur $\NN$ ist eine \begriff{Unterstruktur} von $\MM$ (in Zeichen $\NN \le \MM$), wenn
	\begin{itemize}
		\item $N \subseteq M$
		\item $f^{\NN} = f^{\MM}_{\mid N^{\arity(f)}}$ $\qquad$\selfnote{(insbesondere: $f^{\MM}(N^{\arity(f)} \subseteq N)$)}
		\item $R^{\NN} = R^{\MM} \cap N^{\arity(R)}$
		\item $c^{\NN} = c^{\MM}$
	\end{itemize}
für alle Symbole in $\LL$. Dabei nennt man $\MM$ eine \begriff{Erweiterung} von $\NN$.
\end{definition}
\paragraph{Substructures}[Zoe Chatzidakis]
$N \subseteq M$ gehört zu einer Unterstruktur von $\MM$ genau dann wenn
\begin{itemize}
	\item $N$ erhält alle (Interpretationen von) Konstantensymbolen von $\LL$
	\item $N$ ist abgeschlossen unter Anwendung von (Interpretation  von) von Funktionen von $\LL$.
\end{itemize}
Falls $\LL$ keine Konstantensymbole hat, so ist $\emptyset$ eine Unterstruktur!
\begin{*remark}[Fehm]
	Beachte dabei das Leere Mengen nach \propref{def_1_2_3} keine Unterstruktur sein kann.\\ Betrachten wir konkrete Auswirkungen von Weglassen von Symbolen. Beim Weglassen des neutralen Elements in der Sprache der Gruppen passiert z.B. nichts, weil sowohl neutrales Element als auch Inversion aus der Multiplikation definierbar sind und wir die leere menge nicht zulassen.\\
	Lassen wir die Inversion weg, so müssen Unterstrukturen nicht mehr abgeschlossen unter Inversion sein, sind also nicht notwendigerweise Untergruppen (sondern nur Untermonoide).  
\end{*remark}
\subsection{Formel und Theorien}

\subsection{Kompaktheitssatz}

\subsection{Ultraprodukte}

\subsection{Definierbarkeit und Intepretierbarkeit}

\subsection{Quantorenelimination und Modellvollständigkeit}

\subsection{Modelltheorie von Körpern}
\chapter{Cayley Graphs}

\part*{Anhang}
\addcontentsline{toc}{part}{Anhang}
\appendix

\nocite{*}
%\bibliography{literatur}
\bibliographystyle{acm}

%\printglossary[type=\acronymtype]

\printindex

\end{document}
