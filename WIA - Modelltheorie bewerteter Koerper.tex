\documentclass[ngerman,a4paper,order=firstname]{mathscript}
\usepackage{mathoperators}

\usepackage{xcolor}
% % % local commands
\DeclareMathOperator{\Ad}{Ad}				% Adjoint
\DeclareMathOperator{\PSL}{PSL} 			% projective linear group 
\newcommand{\with}{\text{ with }}
\newcommand{\nd}{\text{ and }}
\renewcommand{\rhd}{\triangleright}
\renewcommand{\lhd}{\triangleleft} 			% normal subgroups
\DeclareMathOperator{\Set}{Set}				% Category of sets
\DeclareMathOperator{\Vect}{Vect}			% Category of vector spaces
\DeclareMathOperator{\Grp}{Grp}				% Category of groups
\DeclareMathOperator{\Mod}{Mod}				% Cat of moduls
\DeclareMathOperator{\Ann}{Ann}				% annihilator
\newcommand{\Circlearrowleft}{\rotatebox{180}{$\circlearrowright$}}
\DeclareMathOperator{\Cl}{Cl}				% conjugation class of something.
\DeclareMathOperator{\Iso}{Iso}

% % % % Model theory
\newcommand{\LL}{\mathcal L}
\newcommand{\RR}{\mathcal R}
\newcommand{\FF}{\mathcal F}
\newcommand{\MM}{\mathcal M} 		
\newcommand{\NN}{\mathcal N}
\newcommand{\CC}{\mathcal C}
\newcommand{\arity}{\lambda}

% % % % better \le and \ge
\renewcommand{\ge}{\geqslant}
\renewcommand{\le}{\leqslant}

\DeclareMathOperator{\group}{group}
\DeclareMathOperator{\ring}{ring}

% % % % color stuff
\newcommand{\selfnote}[1]{\textcolor{gray}{#1}}

% get this stupid arrows:
%\usepackage{mathabx,graphicx}  % ---> add to mathoperators
%\def\Circlearrowleft{\ensuremath{%
%		\rotatebox[origin=c]{180}{$\circlearrowleft$}}}
%\def\Circlearrowright{\ensuremath{%
%		\rotatebox[origin=c]{180}{$\circlearrowright$}}}
%\def\CircleArrowleft{\ensuremath{%
%		\reflectbox{\rotatebox[origin=c]{180}{$\circlearrowleft$}}}}
%\def\CircleArrowright{\ensuremath{%
%		\reflectbox{\rotatebox[origin=c]{180}{$\circlearrowright$}}}}
%\begin{document}
%	\Huge
%	$\circlearrowleft \circlearrowright $
%	
%	$\Circlearrowleft \Circlearrowright $
%	
%	$\CircleArrowleft \CircleArrowright $

% % % local packages
\usepackage{braids}

\newlist{remarkenum}{enumerate}{1}
\setlist[remarkenum]{label=(\alph*),ref=\theremark~(\alph*)}
\crefalias{remarkenumi}{remark}

\newlist{propenum}{enumerate}{1}
\setlist[propenum]{label=(\alph*),ref=\theproposition~(\alph*)}
\crefalias{propenumi}{proposition}

\newlist{expenum}{enumerate}{1}
\setlist[expenum]{label=(\alph*),ref=\theexample~(\alph*)}
\crefalias{expenumi}{example}

\newlist{lemmaenum}{enumerate}{1}
\setlist[lemmaenum]{label=(\alph*),ref=\thelemma~(\alph*)}
\crefalias{lemmaenumi}{lemma}

\newlist{defenum}{enumerate}{1}
\setlist[defenum]{label=(\roman*),ref=\thedefinition~(\roman*)}
\crefalias{defenumi}{definition}

\title{\textbf{WIA - Modelltheorie Bewerteter Körper}}
\author{Dozent: Prof. Dr. \person{Arno Fehm}}

\begin{document}
\pagenumbering{roman}
\pagestyle{plain}

\maketitle

\hypertarget{tocpage}{}
\tableofcontents
\bookmark[dest=tocpage,level=1]{Inhaltsverzeichnis}

\pagebreak
\pagenumbering{arabic}
\pagestyle{fancy}

%\chapter*{Vorwort}
%Dieser Kurs findet im Rahmen des wissenschaftlichen Arbeiten im Master statt. Und im Zuge der Corona Sache auch nur als Reading-Kurs mit Chats im Matrix System. Dazu werden Fragen zum Vorlesungsskript, Übungsaufgaben besprochen. Angedacht ist eigentlich ein Seminarteil, der bisher noch ungeregelt bleibt.

Ja, dann viel Spass bei \ital{Modelltheorie bewerteter Körper}
\chapter{Grundlagen}
\section{Körper}
\subsection{Körpererweiterungen}

\subsection{Galoistheorie}

\subsection{Ringerweiterungen}
\section{Modelltheorie}
\subsection{Spache und Strukturen}
Literatur
\begin{definition}[Sprache]
	Eine \begriff{Sprache} $\LL$ ist Quadrupel $\LL = (\FF,\RR, \CC, \arity)$, wobei $\FF,\RR, \CC$ Mengen sind und $\arity \colon \FF \cup \RR \to \N$.
	\begin{itemize}
		\item Elemente von $\FF,\RR,\CC$ nennen wir Funktionssymbole, Relationssymbole und Konstantensymbole (auch -zeichen)
		\item $\arity(f)$ bzw. $\arity(R)$ gibt die \begriff{Stelligkeit} von $f$ bzw. $R$ an
		\item Kardinalität  von $\LL$ ist $\abs{\LL} := \abs{\FF \cup \RR \cup \CC}$
	\end{itemize}
\end{definition}
\begin{example}
	\begin{itemize}
		\item \emph{Sprache der Gruppen} ist $\LL_{\group} = \set{\ast, \cdot^{-1}}$, wobei $\ast$ ist zweistellig und $\cdot^{-1}$ is einstelliges Funktionssymbol.
		\item \emph{Sprache der Ringe} ist $\LL_{\ring} = \set{+, \cdot, 0, 1}$, dabei $+$ und $\cdot$ zweistellig Funktionssymbole und 0,1 Konstantensymbole.
	\end{itemize}
\end{example}
\subsection{Formel und Theorien}

\subsection{Kompaktheitssatz}

\subsection{Ultraprodukte}

\subsection{Definierbarkeit und Intepretierbarkeit}

\subsection{Quantorenelimination und Modellvollständigkeit}

\subsection{Modelltheorie von Körpern}
\chapter{Cayley Graphs}

\part*{Anhang}
\addcontentsline{toc}{part}{Anhang}
\appendix

\nocite{*}
%\bibliography{literatur}
\bibliographystyle{acm}

%\printglossary[type=\acronymtype]

\printindex

\end{document}
