\section{Modelltheorie}
\subsection{Spache und Strukturen}
Literatur
\begin{definition}[Sprache]
	Eine \begriff{Sprache} $\LL$ ist Quadrupel $\LL = (\FF,\RR, \CC, \arity)$, wobei $\FF,\RR, \CC$ Mengen sind und $\arity \colon \FF \cup \RR \to \N$.
	\begin{itemize}
		\item Elemente von $\FF,\RR,\CC$ nennen wir Funktionssymbole, Relationssymbole und Konstantensymbole (auch -zeichen)
		\item $\arity(f)$ bzw. $\arity(R)$ gibt die \begriff{Stelligkeit} von $f$ bzw. $R$ an
		\item Kardinalität  von $\LL$ ist $\abs{\LL} := \abs{\FF \cup \RR \cup \CC}$
	\end{itemize}
\end{definition}
\begin{example}
	\begin{itemize}
		\item \emph{Sprache der Gruppen} ist $\LL_{\group} = \set{\ast, \cdot^{-1}}$, wobei $\ast$ ist zweistellig und $\cdot^{-1}$ is einstelliges Funktionssymbol.
		\item \emph{Sprache der Ringe} ist $\LL_{\ring} = \set{+, \cdot, 0, 1}$, dabei sind $+$ und $\cdot$ zweistellig Funktionssymbole und 0,1 Konstantensymbole.
	\end{itemize}
\end{example}
\begin{definition}[$\LL$-Struktur]
	\proplbl{def_1_2_3}
	Sei $\LL = (\FF,\RR, \CC, \arity)$ eine \emph{Sprache}. Eine \begriff{$\LL$-Struktur} ist ein Tupel
	\begin{align*}
		\MM = (M, (f^{\MM})_{f \in \FF}, (R^{\MM})_{R \in \RR}, (c^{\MM})_{c \in \CC}),
	\end{align*}
	wobei
	\begin{itemize}
		\item $M$ \emph{nichtleere} Menge
		\item $f^{\MM}\colon M^{\arity(f)} \to M$
		\item $R^{\MM} \subseteq M^{\arity(R)}$
		\item $c^{\MM} \in M$
	\end{itemize}
	die \emph{Interpretation} der entsprechenden Symbole sind.
	\item \emph{Kardinalität} von $\MM$ ist $\abs{\MM} = \abs{M}$
\end{definition}
\begin{*remark}
	\marganote{Wir werden lieber $R^{\MM}(a)$ für $\underline{a} \in R^{\MM}$ schreiben. ($\underline{a} \in M^{\arity(R)}$) Hierbei ist $\underline{a} = (a_1, \dots, a_n)$!}
\end{*remark}
\begin{example}
	Jeder Ring ist auf natürliche Weise eine $\LL_{\ring}$-Struktur.
\end{example}
\begin{definition}[Unterstruktur]
	Sei $\MM$ eine $\LL$-Struktur. Eine $\LL$-Struktur $\NN$ ist eine \begriff{Unterstruktur} von $\MM$ (in Zeichen $\NN \le \MM$), wenn
	\begin{itemize}
		\item $N \subseteq M$
		\item $f^{\NN} = f^{\MM}_{\mid N^{\arity(f)}}$ $\qquad$\marganote{(insbesondere: $f^{\MM}(N^{\arity(f)} \subseteq N)$)}
		\item $R^{\NN} = R^{\MM} \cap N^{\arity(R)}$
		\item $c^{\NN} = c^{\MM}$
	\end{itemize}
für alle Symbole in $\LL$. Dabei nennt man $\MM$ eine \begriff{Erweiterung} von $\NN$.
\end{definition}
\paragraph{Substructures}[Zoe Chatzidakis]
$N \subseteq M$ gehört zu einer Unterstruktur von $\MM$ genau dann wenn
\begin{itemize}
	\item $N$ erhält alle (Interpretationen von) Konstantensymbolen von $\LL$
	\item $N$ ist abgeschlossen unter Anwendung von (Interpretation  von) von Funktionen von $\LL$.
\end{itemize}
Falls $\LL$ keine Konstantensymbole hat, so ist $\emptyset$ eine Unterstruktur!
\begin{*remark}[Fehm]
	Beachte dabei das Leere Mengen nach \propref{def_1_2_3} keine Unterstruktur sein kann.\\ Betrachten wir konkrete Auswirkungen von Weglassen von Symbolen. Beim Weglassen des neutralen Elements in der Sprache der Gruppen passiert z.B. nichts, weil sowohl neutrales Element als auch Inversion aus der Multiplikation definierbar sind und wir die leere menge nicht zulassen.\\
	Lassen wir die Inversion weg, so müssen Unterstrukturen nicht mehr abgeschlossen unter Inversion sein, sind also nicht notwendigerweise Untergruppen (sondern nur Untermonoide).
	Es gab dazu auch auch einen coolen Artikel, eventuell hier als Literatur einbinden? 
\end{*remark}
\begin{definition}[$\LL$-Morphismus]
	Seien $\NN, \MM$ $\LL$-Strukturen. Eine Abbildung
	\begin{align*}
		\xi \colon M \to N
	\end{align*}
	ist \begriff{$\LL$-Morphismus}, wenn 
	\begin{itemize}
		\item $\xi(f^{\MM}(\underline{a})) = f^{\NN}(\xi(\underline{a}))$ für $\underline{a} \in M^{\arity(f)}$
		\item $R^{\N}(\underline{a}) \implies R^{\NN}(\xi(\underline{a}))$ für $\underline{a} \in M^{\arity(f)}$   $\marganote{R^{\NN}\subseteq \xi(R^{\NN})}$
		\item $\xi(C^{\MM}) = c^{\MM}$
	\end{itemize}
	für alle Symbole $f \in \FF, R \in \RR, c \in \CC$. \fehmnote{(Hier schreiben wir $\xi(f(a_1, \dots, a_n))$ für $(\xi(a_1),\dots, \xi(a_n)$)}. $\xi$ ist eine \begriff{$\LL$}-Einbettung, wenn $\xi$ zusätzlich injektiv ist und 
	\begin{align*}
		R^{\MM}(\ul{a}) \iff R^{\NN}(\xi(\ul{a})) \quad \forall \ul{a}\in M^{\arity(R)}\quad \marganote{R^{\MM} = \xi(R^{\NN})}
	\end{align*}
	Ein \begriff{$\LL$-Isomorphismus} ein bijektiver $\LL$-Morphismus, dessen Umkehrabbildung wieder ein $\LL$-Morphismus ist. Wir schreiben $\MM \cong \NN$, wenn es einen $\LL$-Isomorphismus $\MM \to \NN$ gibt.
\end{definition}
\begin{remark}
	Genau dann ist $\MM \le \NN$, wenn $M \hookrightarrow N$ eine $\LL$-Einbettung ist.
\end{remark}
\begin{definition}[Erweiterung]
	Seien $\LL$ und $\LL'$ Sprachen. Wir nennen $\LL'$ eine Erweiterung, wenn
	\begin{align*}
		\FF \subseteq \FF', \R \subseteq \RR', \CC \subseteq \CC' \nd \arity'_{\mid \FF \cup \RR} = \arity
	\end{align*}
	Sei $\MM'$ eine $\LL'$-Struktur. Das \begriff{$\LL$-Redukt} von $\MM'$ ist
	\begin{align*}
		\MM'_{\mid \LL} := (M', (f^{\MM})_{f \in \FF}, (R^{\MM}_{R \in \R},(c^{\MM}_{c \in \CC}))
	\end{align*}
\end{definition}
\begin{example}
	Sei $\MM$ $\LL$-Struktur. Für $C \subseteq M$ bezeichne $\LL(C) = (\FF,\RR, \CC \cup C,\arity)$ die Erweiterung von $\LL$ um Konstantensymbole aus $C$. Die $\LL$-Struktur $\MM$ ist dann auf natürliche Weise eine $\LL(C)$-Struktur.
\end{example}
\begin{example}
	Sind $(\MM_i)_{i \in I}$ $\LL$-Strukturen, so wir das kartesische Produkt $\MM := \prod_{i \in I}M_i$ zu einer $\LL$-Struktur durch
	\begin{itemize}
		\item $f^{\MM}((\ul{a}_i)_{i \in I}) = (f^{\MM_i}(\ul{a}_i))_{i \in I}$
		\item $R^{\MM}\prod_{i \in I} R^{\MM_i}$
		\item $c^{\MM} = (c^{\MM_i})_{i \in I}$
	\end{itemize}
\end{example}
\begin{definition}[mehrsortig]
	Enthält eine Sprache $\LL$ einstellige Relationssymbole $S_1, \dots, S_n$, so nennen wir eine $\LL$-Struktur $\MM$ \begriff{mehrsortig}, wenn
	\begin{align*}
		M = S_1^{\MM} \discup \dots \discup S_n^{\MM}
		\intertext{und schreibe dies dann auch als}
		\MM = (S_1^{\MM}, \dots, S_n^{\MM}, \dots)
	\end{align*}
	Per Konvention betrachtet man in mehrsortigen Strukturen die Symbole immer als bestimmte Sorten zugehörig, schreibt man etwa
	\begin{align*}
		f^{\MM}\colon S_1^{\MM} \to S_2^{\MM}
	\end{align*}
	und ignoriert da andere Sorten.
\end{definition}
\begin{example}
	Die mehrsortigen Strukturen einer Sprache $\LL = (S_1, S_2, S_3)$, wobei ``$\epsilon \subseteq S_1 \times S_2$'' können als Mengen $S_1$ mit einer Familie $S_2$ von Teilmengen von $S_1$ aufgebaut werden.
\end{example}
\subsection{Formel und Theorien}
\begin{definition}[$\LL$-Term, Interpretation]
	Ein \begriff{$\LL$-Term} ist eine Zeichenkette der Form $x_i$ (eine Variable), $c$ (ein Konstantensymbol) oder $f(t_1, \dots, t_{\arity(f)})$, wobei $f$ ein Funktionssymbol und $t_1, \dots, t_{\arity(f)}$ Terme. Die \begriff{Interpretation} eine $\LL$-Terms $t = t(x_1,\dots, x_n)$ in Variablen $x_1, \dots, x_n$ in einer $\LL$-Struktur $\MM$ ist die entsprechende Abbildung $t^{\MM}\colon M^n \to M$.
\end{definition}
\subsection{Kompaktheitssatz}

\subsection{Ultraprodukte}

\subsection{Definierbarkeit und Intepretierbarkeit}

\subsection{Quantorenelimination und Modellvollständigkeit}

\subsection{Modelltheorie von Körpern}