\section{Modelltheorie}
\subsection{Spache und Strukturen}
Literatur
\begin{definition}[Sprache]
	Eine \begriff{Sprache} $\LL$ ist Quadrupel $\LL = (\FF,\RR, \CC, \arity)$, wobei $\FF,\RR, \CC$ Mengen sind und $\arity \colon \FF \cup \RR \to \N$.
	\begin{itemize}
		\item Elemente von $\FF,\RR,\CC$ nennen wir Funktionssymbole, Relationssymbole und Konstantensymbole (auch -zeichen)
		\item $\arity(f)$ bzw. $\arity(R)$ gibt die \begriff{Stelligkeit} von $f$ bzw. $R$ an
		\item Kardinalität  von $\LL$ ist $\abs{\LL} := \abs{\FF \cup \RR \cup \CC}$
	\end{itemize}
\end{definition}
\begin{example}
	\begin{itemize}
		\item \emph{Sprache der Gruppen} ist $\LL_{\group} = \set{\ast, \cdot^{-1}}$, wobei $\ast$ ist zweistellig und $\cdot^{-1}$ is einstelliges Funktionssymbol.
		\item \emph{Sprache der Ringe} ist $\LL_{\ring} = \set{+, \cdot, 0, 1}$, dabei $+$ und $\cdot$ zweistellig Funktionssymbole und 0,1 Konstantensymbole.
	\end{itemize}
\end{example}
\subsection{Formel und Theorien}

\subsection{Kompaktheitssatz}

\subsection{Ultraprodukte}

\subsection{Definierbarkeit und Intepretierbarkeit}

\subsection{Quantorenelimination und Modellvollständigkeit}

\subsection{Modelltheorie von Körpern}